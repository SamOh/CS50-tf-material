\documentclass{article}
%%%%%%%%%%%%%%%%%%%%%%%%%%%%%%%%%%%%%%%%%%%%%%%%%%%%%%%%%%%%%
% Lecture Specific Information to Fill Out
%%%%%%%%%%%%%%%%%%%%%%%%%%%%%%%%%%%%%%%%%%%%%%%%%%%%%%%%%%%%%
\newcommand{\LectureTitle}{Week 5}
%\newcommand{\LectureDate}{\today}
\newcommand{\LectureDate}{October\ 4,\ 2016}
\newcommand{\LectureClassName}{CS\ 50}
\newcommand{\LatexerName}{Samuel\ Oh}
%%%%%%%%%%%%%%%%%%%%%%%%%%%%%%%%%%%%%%%%%%%%%%%%%%%%%%%%%%%%%

% Change "article" to "report" to get rid of page number on title page
\usepackage{amsmath,amsfonts,amsthm,amssymb}
\usepackage{setspace}
\usepackage{Tabbing}
\usepackage{fancyhdr}
\usepackage{hyperref}
\usepackage{lastpage}
\usepackage{extramarks}
\usepackage{chngpage}
\usepackage{soul,color}
\usepackage{graphicx,float,wrapfig}
\usepackage{afterpage}
\usepackage{abstract}
\usepackage{tikz}
\usepackage{ulem}
\let\underbar\uline
% In case you need to adjust margins:
\topmargin=-0.45in
\evensidemargin=0in
\oddsidemargin=0in
\textwidth=6.5in
\textheight=9.0in
\headsep=0.25in

% Setup the header and footer
\pagestyle{fancy}
\lhead{\LatexerName}
\chead{\LectureClassName: \LectureTitle}
\rhead{\LectureDate}
\lfoot{\lastxmark}
\cfoot{}
\rfoot{Page\ \thepage\ of\ \pageref{LastPage}}
\renewcommand\headrulewidth{0.4pt}
\renewcommand\footrulewidth{0.4pt}

%%%%%%%%%%%%%%%%%%%%%%%%%%%%%%%%%%%%%%%%%%%%%%%%%%%%%%%%%%%%%
% Some tools
\newcommand{\enterTopicHeader}[1]{\nobreak\extramarks{#1}{#1 continued on next page\ldots}\nobreak
                                    \nobreak\extramarks{#1 (continued)}{#1 continued on next page\ldots}\nobreak}
\newcommand{\exitTopicHeader}[1]{\nobreak\extramarks{#1 (continued)}{#1 continued on next page\ldots}\nobreak
                                   \nobreak\extramarks{#1}{}\nobreak}

\newlength{\labelLength}
\newcommand{\labelAnswer}[2]
  {\settowidth{\labelLength}{#1}
   \addtolength{\labelLength}{0.25in}
   \changetext{}{-\labelLength}{}{}{}
   \noindent\fbox{\begin{minipage}[c]{\columnwidth}#2\end{minipage}}
   \marginpar{\fbox{#1}}

   % We put the blank space above in order to make sure this
   % \marginpar gets correctly placed.
   \changetext{}{+\labelLength}{}{}{}}

\setcounter{secnumdepth}{0}
\newcommand{\TopicName}{}
\newcounter{TopicCounter}
\newenvironment{Topic}[1][Problem \arabic{TopicCounter}]
  {\stepcounter{TopicCounter}
   \renewcommand{\TopicName}{#1}
   \section{\TopicName}
   \enterTopicHeader{\TopicName}}
  {\exitTopicHeader{\TopicName}}

\setcounter{secnumdepth}{0}
\newcommand{\ExampleSectionName}{}
\newcounter{ExampleSectionCounter}[TopicCounter]
\newenvironment{ExampleSection}[1][Example \arabic{ExampleSectionCounter}]
  {\stepcounter{ExampleSectionCounter}
   \renewcommand{\ExampleSectionName}{#1}
   \section{\ExampleSectionName}
   \enterTopicHeader{\ExampleSectionName}}
  {\exitTopicHeader{\ExampleSectionName}}

\setcounter{secnumdepth}{0}
\newcounter{ExampleBoxCounter}[TopicCounter]
\newcommand{\examplebox}[1]
  {
  % We put this space here to make sure we're disconnected from the previous
   % passage
   \stepcounter{ExampleBoxCounter}
   \noindent\fbox{\begin{minipage}[c]{\columnwidth}#1\end{minipage}}\enterTopicHeader{\ExampleSectionName}\exitTopicHeader{\ExampleSectionName}\marginpar{\fbox{\#\arabic{ExampleBoxCounter}}}
   % We put the blank space above in order to make sure this
   % \marginpar gets correctly placed.
   \vskip10pt
   }

\renewcommand{\contentsname}{{\normalsize Topics Covered}}
\renewcommand{\abstractname}{\LectureTitle\ Summary}
\renewcommand{\absnamepos}{flushleft}

\theoremstyle{definition}
\newtheorem*{dfn}{Definition}
%----------------------------------------------------------------------------------------
%	CUSTOM COMMANDS DUE TO LAZINESS
%----------------------------------------------------------------------------------------

%--------------------------------------------
% Numbers cuz reasons
\newcommand{\C}{{\mathbb C}}
\newcommand{\Q}{{\mathbb Q}}
\newcommand{\R}{{\mathbb R}}
\newcommand{\N}{{\mathbb N}}
\newcommand{\Z}{{\mathbb Z}}
\newcommand{\Sp}{{\mathbb S}}
\newcommand{\A}{{\mathbb A}}

%--------------------------------------------
% Symbols are long
\newcommand{\ph}{{\Phi}}
\newcommand{\vp}{{\varphi}}
\newcommand{\ra}{{\rightarrow}}
\newcommand{\mt}{{\mapsto}}
% Independence
\newcommand\ind{\protect\mathpalette{\protect\independenT}{\perp}}
\def\independent#1#2{\mathrel{\rlap{$#1#2$}\mkern2mu{#1#2}}}

%--------------------------------------------
% Partial derivatives are hard
\newcommand{\p}[2]{\frac{\partial #1}{\partial #2}}
\newcommand{\pp}[2]{\frac{\partial^2 {#1}} {\partial {#2} ^2}}
\newcommand{\pmix}[3]{\frac{\partial^2 {#1}}{\partial {#2}\, \partial {#3}}}

%----------------------------------------------------------------------------------------
%	COMMANDS THAT I NEVER REMEMBER LEL
%----------------------------------------------------------------------------------------

% Integrals
% \int_{a}^{b} [insert integrand] [insert differential]

% Double integral
% \iint_{Values being integrated over} [insert integrand] [\, differential] [\, differential]

% Multi integral
% \idotsint_{Values} [insert integrand f(x, \dots, x_n)] [\, differential \dots differential]

% Summation
% \sum_{n = start value}^{end value} [function]

% Limits
% \lim_{x \to value} [function]

% I.I.D.
% $\overset{\text{i.i.d.}}{\sim}$

% Gradient
% \nabla (\nabla deez nuts)

% Cases
% \begin{cases}
%	{case 1} & \text{if \,} {condition} \\
%	{case 2} & \text{if \,} {condition} \\
% \end{cases}

\begin{document}
\begin{spacing}{1.1}
\newpage

\tableofcontents
\addtocontents{toc}{~\hfill\textbf{Page}\par}
\vskip10pt
\hrule
\vskip10pt

% When topics are long, it may be desirable to put a \newpage or a
% \clearpage before each Topic environment
% \newpage
\begin{Topic}[Structs]
A structure is basically a super-variable. We use structures to logically group
together a collection of elements (which we will call fields) that together comprise
 a larger element. For example, here’s the potential definition of a structure
(struct) for a student:
\begin{verbatim}
struct student_t {
  string name;
  int year;
  float gpa;
};
\end{verbatim}
Once we’ve defined the structure, we can then use it like any other type. We access fields of the structure with the
dot operator (.). We can also have pointers to the structure like we can have pointers to any other type. If we have
a pointer to a structure, we access its fields with the arrow operator ($->$).
\begin{verbatim}
struct student_t john;
struct student_t * mary = malloc(sizeof(struct student_t));
john.name = "John";
mary->name = "Mary";
john.year = 2014;
mary->year = 2015;
john.gpa = 3.48;
mary->gpa = 3.55;
\end{verbatim}
\end{Topic}
\begin{Topic}[Linked Lists]
How do you create a linked list? \newline
Insert into a linked list (at head, tail, middle)? \newline
Delete out of a linked list? \newline
Delete an entire linked list? \newline
And iterate over a linked list?
\begin{verbatim}
typedef struct node
{
    // just some form of data; could be a char* or whatever
    int i;

    // pointer to next node; have to include `struct` since this is a recursive definition
    struct node *next;

    node;
}
\end{verbatim}

How do you alter and iterate over linked lists?
The end of a linked list will always be marked by a pointer to NULL. To iterate
over a list, then, use a $node *head_ptr$ and simply loop through the list until its
next field points to NULL. Similarly, to insert or delete a node, simply change
where the $node *next$ points. Be careful about the order of operations, though!
Especially when inserting, if done in the wrong order you could lose track of your list!

\end{Topic}
\begin{Topic}[Hash Tables]
Hash tables are a great data structure to store information in a semi-organized way. A hash table is made up of at
least two parts: (1) an array and (2) a hash function. More advanced hash tables, which we will be using, also make
use of (3) linked lists. Say we want to hash some strings. Let’s create a hash table of linked lists nodes, where each
of those nodes contains a string and a pointer to another node.
\begin{verbatim}
typedef struct _node {
  char word[50]; // max length of a word is 50
  struct _node *next;
} node;

node *hashtable[3]; // our hashtable has 3 possible hash values
\end{verbatim}
We can then hash a string to compute a hash code, which is entirely derived from the string itself. For example,
hash("hello") may be 298. We then mod that to be in the range [0, 2] (for our hashtable) and insert it.
\begin{verbatim}
[0]
[1] -> "hello"
[2]
\end{verbatim}
Then, if we hash("world") and we get 325, we can mod that to be in the range [0, 2] and insert it, rearranging the
nodes of the linked list as necessary so we don’t lose what was there before.
\begin{verbatim}
[0]
[1] -> "world" -> "hello"
[2]
\end{verbatim}
We can look up strings easily by hashing them again, and then searching through the list pointed to by that hash
code to find whether or not they are in the list.
\end{Topic}
\begin{Topic}[Binary Trees and Tries]
Notice how with hash tables we are combining simple data structures (arrays and linked lists) in a way so as to create
a new data structures that has an even higher degree of utility. A trie, discussed in just a moment, uses those same two
simple structures in a different way to do something completely different.
A tree is a data structure very similar to a linked list, but with a different interpretation. The rules of a tree
(into which category binary trees and tries both fall) are as follows: \newline \newline
1. A tree node can point at its children (named left and right, in a binary tree), or at NULL. \newline
2. A tree node may not point at any other node other than those listed above, including itself. \newline \newline
We typically visualize binary trees upside-down from how a regular tree looks.
The root of the tree is the node at the very top. The leaves are the nodes where both
children point to NULL (indicated here as colored-in circles). The branches are the
nodes that are not leaves or the root. And what is a trie, then? A trie is simply a
tree with an arbitrary number of children. This means that a binary tree is simply a
special case of a trie, where the number of children is 2. Let’s declare three structures.
One for a binary tree, one for a binary tree with trie syntax, and one for a trie with
6 children.
\begin{verbatim}
// Binary tree
typedef struct _btree
{
bool valid;
struct _btree *left;
struct _btree *right;
}
btree_t;

// Binary trie
typedef struct _btrie
{
  bool valid;
  struct _btrie *kids[2];
}
btrie_t;

// 6-child trie
typedef struct _trie6
{
  bool valid;
  struct _trie6 *kids[6];
}
trie6_t;
\end{verbatim}
\end{Topic}
\begin{Topic}[Huffman Encoding]
Take notes below from examples in class from study.cs50.net!
\end{Topic}
% Use \Roman{TopicCounter} for each new topic so it increments in the table of contents
% Use \begin{ExampleSection} and \examplebox{} as is necessary

\end{spacing}
\end{document}
