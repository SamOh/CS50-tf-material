\documentclass{article}
%%%%%%%%%%%%%%%%%%%%%%%%%%%%%%%%%%%%%%%%%%%%%%%%%%%%%%%%%%%%%
% Lecture Specific Information to Fill Out
%%%%%%%%%%%%%%%%%%%%%%%%%%%%%%%%%%%%%%%%%%%%%%%%%%%%%%%%%%%%%
\newcommand{\LectureTitle}{Week 3}
%\newcommand{\LectureDate}{\today}
\newcommand{\LectureDate}{September\ 20,\ 2016}
\newcommand{\LectureClassName}{CS\ 50}
\newcommand{\LatexerName}{Samuel\ Oh}
%%%%%%%%%%%%%%%%%%%%%%%%%%%%%%%%%%%%%%%%%%%%%%%%%%%%%%%%%%%%%

% Change "article" to "report" to get rid of page number on title page
\usepackage{amsmath,amsfonts,amsthm,amssymb}
\usepackage{setspace}
\usepackage{Tabbing}
\usepackage{fancyhdr}
\usepackage{hyperref}
\usepackage{lastpage}
\usepackage{extramarks}
\usepackage{chngpage}
\usepackage{soul,color}
\usepackage{graphicx,float,wrapfig}
\usepackage{afterpage}
\usepackage{abstract}
\usepackage{tikz}
\usepackage{ulem}
\let\underbar\uline
% In case you need to adjust margins:
\topmargin=-0.45in
\evensidemargin=0in
\oddsidemargin=0in
\textwidth=6.5in
\textheight=9.0in
\headsep=0.25in

% Setup the header and footer
\pagestyle{fancy}
\lhead{\LatexerName}
\chead{\LectureClassName: \LectureTitle}
\rhead{\LectureDate}
\lfoot{\lastxmark}
\cfoot{}
\rfoot{Page\ \thepage\ of\ \pageref{LastPage}}
\renewcommand\headrulewidth{0.4pt}
\renewcommand\footrulewidth{0.4pt}

%%%%%%%%%%%%%%%%%%%%%%%%%%%%%%%%%%%%%%%%%%%%%%%%%%%%%%%%%%%%%
% Some tools
\newcommand{\enterTopicHeader}[1]{\nobreak\extramarks{#1}{#1 continued on next page\ldots}\nobreak
                                    \nobreak\extramarks{#1 (continued)}{#1 continued on next page\ldots}\nobreak}
\newcommand{\exitTopicHeader}[1]{\nobreak\extramarks{#1 (continued)}{#1 continued on next page\ldots}\nobreak
                                   \nobreak\extramarks{#1}{}\nobreak}

\newlength{\labelLength}
\newcommand{\labelAnswer}[2]
  {\settowidth{\labelLength}{#1}
   \addtolength{\labelLength}{0.25in}
   \changetext{}{-\labelLength}{}{}{}
   \noindent\fbox{\begin{minipage}[c]{\columnwidth}#2\end{minipage}}
   \marginpar{\fbox{#1}}

   % We put the blank space above in order to make sure this
   % \marginpar gets correctly placed.
   \changetext{}{+\labelLength}{}{}{}}

\setcounter{secnumdepth}{0}
\newcommand{\TopicName}{}
\newcounter{TopicCounter}
\newenvironment{Topic}[1][Problem \arabic{TopicCounter}]
  {\stepcounter{TopicCounter}
   \renewcommand{\TopicName}{#1}
   \section{\TopicName}
   \enterTopicHeader{\TopicName}}
  {\exitTopicHeader{\TopicName}}

\setcounter{secnumdepth}{0}
\newcommand{\ExampleSectionName}{}
\newcounter{ExampleSectionCounter}[TopicCounter]
\newenvironment{ExampleSection}[1][Example \arabic{ExampleSectionCounter}]
  {\stepcounter{ExampleSectionCounter}
   \renewcommand{\ExampleSectionName}{#1}
   \section{\ExampleSectionName}
   \enterTopicHeader{\ExampleSectionName}}
  {\exitTopicHeader{\ExampleSectionName}}

\setcounter{secnumdepth}{0}
\newcounter{ExampleBoxCounter}[TopicCounter]
\newcommand{\examplebox}[1]
  {
  % We put this space here to make sure we're disconnected from the previous
   % passage
   \stepcounter{ExampleBoxCounter}
   \noindent\fbox{\begin{minipage}[c]{\columnwidth}#1\end{minipage}}\enterTopicHeader{\ExampleSectionName}\exitTopicHeader{\ExampleSectionName}\marginpar{\fbox{\#\arabic{ExampleBoxCounter}}}
   % We put the blank space above in order to make sure this
   % \marginpar gets correctly placed.
   \vskip10pt
   }

\renewcommand{\contentsname}{{\normalsize Topics Covered}}
\renewcommand{\abstractname}{\LectureTitle\ Summary}
\renewcommand{\absnamepos}{flushleft}

\theoremstyle{definition}
\newtheorem*{dfn}{Definition}
%----------------------------------------------------------------------------------------
%	CUSTOM COMMANDS DUE TO LAZINESS
%----------------------------------------------------------------------------------------

%--------------------------------------------
% Numbers cuz reasons
\newcommand{\C}{{\mathbb C}}
\newcommand{\Q}{{\mathbb Q}}
\newcommand{\R}{{\mathbb R}}
\newcommand{\N}{{\mathbb N}}
\newcommand{\Z}{{\mathbb Z}}
\newcommand{\Sp}{{\mathbb S}}
\newcommand{\A}{{\mathbb A}}

%--------------------------------------------
% Symbols are long
\newcommand{\ph}{{\Phi}}
\newcommand{\vp}{{\varphi}}
\newcommand{\ra}{{\rightarrow}}
\newcommand{\mt}{{\mapsto}}
% Independence
\newcommand\ind{\protect\mathpalette{\protect\independenT}{\perp}}
\def\independent#1#2{\mathrel{\rlap{$#1#2$}\mkern2mu{#1#2}}}

%--------------------------------------------
% Partial derivatives are hard
\newcommand{\p}[2]{\frac{\partial #1}{\partial #2}}
\newcommand{\pp}[2]{\frac{\partial^2 {#1}} {\partial {#2} ^2}}
\newcommand{\pmix}[3]{\frac{\partial^2 {#1}}{\partial {#2}\, \partial {#3}}}

%----------------------------------------------------------------------------------------
%	COMMANDS THAT I NEVER REMEMBER LEL
%----------------------------------------------------------------------------------------

% Integrals
% \int_{a}^{b} [insert integrand] [insert differential]

% Double integral
% \iint_{Values being integrated over} [insert integrand] [\, differential] [\, differential]

% Multi integral
% \idotsint_{Values} [insert integrand f(x, \dots, x_n)] [\, differential \dots differential]

% Summation
% \sum_{n = start value}^{end value} [function]

% Limits
% \lim_{x \to value} [function]

% I.I.D.
% $\overset{\text{i.i.d.}}{\sim}$

% Gradient
% \nabla (\nabla deez nuts)

% Cases
% \begin{cases}
%	{case 1} & \text{if \,} {condition} \\
%	{case 2} & \text{if \,} {condition} \\
% \end{cases}

\begin{document}
\begin{spacing}{1.1}
\newpage

\tableofcontents
\addtocontents{toc}{~\hfill\textbf{Page}\par}
\vskip10pt
\hrule
\vskip10pt

% When topics are long, it may be desirable to put a \newpage or a
% \clearpage before each Topic environment
% \newpage
\begin{Topic}[CS50 Study]
	The material for this week may get a little confusing. Make sure to use your resources,
	such as CS50 Study\url{https://study.cs50.net/binary_search?toc=binary_search,bubble_sort,insertion_sort,selection_sort,merge_sort}
\end{Topic}
% Use \Roman{TopicCounter} for each new topic so it increments in the table of contents
% Use \begin{ExampleSection} and \examplebox{} as is necessary
\begin{Topic}[Arrays Review]
	What data types can you use arrays for? \\\\
	Do you need a \textbackslash0 after every type of array?
\end{Topic}
\begin{Topic}[Asymptotic Notation]
	We measure computational complexity (aka "Big O") by comparing how quickly the number
	of operations ittakes to complete the function or program grows as we increase the size
	of the input to some arbitrarily largevalue.  This lets us know how the well the algorithm
	scales with larger problems.  When we talk about "inputsize", we mean whatever makes sense
	with respect to the algorithm.  (For a string, probably the length of the string; for an array,
	probably the number of elements in the array...)
	Just like the upper bound ("big O") we also pay attention to the lower bound $\Omega$, on an algorithm's
	running time. The lower bound corresponds to an algorithm's best case scenario. \newline
	Some examples of Upper/Lower bounds: \newline \newline
	O(n), O(1), O(n\textsuperscript{2}), O(log(n)), O(n log(n))
\end{Topic}
	\begin{Topic}[Sorting Algorithms]
	\textbf{Example 1: Linear Search} \newline
	What is the upper bound O, and the lowerbound $\Omega$? \newline
	Hint: Think back to David's phone book example from class. \newline \newline \newline \newline
	When you perform a linear search in the phone book, it could be on the first page, and therefore, it
	would only take one operation. It could also be on the last page, in which case it would take
	n operations. \newline \newline
	\textbf{Example 2: Binary Search} \newline
	Binary search is an algorithm than runs, in the worst case, in O(log n) time and
	in the best case, Ω(1) time. \newline
	Think back to the phone book example David did in class: if you get lucky and the
	first page you choose includes the person you’re looking for, that took only 1 operation.
	Otherwise, you will have to do max log(n) operations to find them. \newline \newline
	This only works if the list you’re looking through is sorted. Otherwise, it would
	be impossible to know for certain that the halves of the list you’re discarding don’t
	have what you’re looking for.
	to free variables, we have \newline \newline
	\textbf{Example 3: Bubble Sort}
	Bubble Sort runs in O(n2) time complexity. It works on an array of size n by
	iterating across the unsorted part of the array, swapping adjacent items that are
	out of place. In this way, larger elements tend to bubble to the top. \newline \newline
	For example, let’s say you’re given array [4, 1, 7, 10, 3]: what are the steps?\newline \newline \newline \newline
	The lower-bound on runtime is Ω(n), though this requires us to use a swap counter
	or flag to know if we didn’t make any swaps (in which case the array is already sorted!) \newline
	Why do you need to know how many swaps were made?\newline \newline
	Can you explain the math behind the runtime? \newline \newline
	For practice, write out the pseudocode! \newline \newline \newline
	\textbf{Example 4: Selection Sort}
	Selection sort runs in O(n2) time complexity. It works on an array of size n,
	building the final sorted array by removing, one at a time, the smallest element
	in the unsorted array and placing it at the head of the sorted array. \newline
	For example, with array [50, 1, 42, 4, 51]: \newline \newline \newline

	The lower-bound on runtime is Ω(n2). Selection sort is dumb, and always will have
	to find the smallest element on each pass.\newline \newline
	\textbf{Example 5: Insertion Sort}
	Insertion sort runs in O(n2) time complexity. It works on an array of size n,
	building the final sorted array one element at a time by removing, one at a time,
	elements in the unsorted portion of the array and placing them in their correct
	position in the sorted array. \newline
	For example, with array [2, 8, 1, 4, 3]: \newline \newline \newline
	Insertion sort is in many cases better than bubble and selection Sort. Whereas
	both bubble and selection sort in normal cases, without any optimizations, run
	in the best case n2, insertion sort will run in n. But the upper bound on runtime is nevertheless n2.

	The lower-bound on runtime is Ω(n2). Selection sort is dumb, and always will have
	to find the smallest element on each pass. \newline	\newline
	\textbf{Example 6: Merge Sort}
	Merge sort is very different from the previous three sorting algorithms. It is
	most easily implemented using recursion instead of iteration

	If worried your students might be uncomfortable with recursion, here’s a sample
	program showing both an iterative and recursive approach to summing integers.

	And second, it runs in $nlog(n)$ time making it significantly faster than the other
	algorithms. Its power comes from its divide and conquer approach. Think of it
	like this: whereas the other sorting algorithms take n comparisons n times as they
	traverse an array, merge sort only has to do n comparisons log(n) times since the
	size of the array is halved each recursive call (think back to the phone book from binary search!) \newline
	Let's try an example with array, [3,5,2,6,4,1]: \newline \newline \newline
	Below is the code for Merge sort.
	\begin{verbatim}
	int mergesort(int x, int y[])
	{
    if (x == 0)
        return y[0];
    else
        return mergesort(x - 1, y) + y[x];
	}
	\end{verbatim}

	\end{Topic}
	\begin{Topic}[Algorithms Summary]
		\begin{table}[h]
		\centering
		\caption{Runtime Summaries}
		\label{my-label}
		\begin{tabular}{|p{3cm}|p{10cm}|p{1cm}|p{1cm}|}
		\hline
		Algorithm Name & Basic Concept                                                                                                                               & O                      & $\Omega$               \\ \hline
		Selection Sort & Find the \textbf\{smallest\} unsorted element in an array and swap it with the first unsorted element of that array.                        & n\textsuperscript{2}   & n\textsuperscript{2}   \\ \hline
		Bubble Sort    & Swap adjacent pairs of elements if they are out of order, effectively “bubbling” larger elements to the right and smaller ones to the left. & n\textsuperscript{2}   & n                      \\ \hline
		Insertion Sort & Proceed once through the array from left-to-right, shifting elements as necessary to insert each element into its correct place.            & n\textsuperscript{2}   & n                      \\ \hline
		Merge Sort     & Split the full array into subarrays, then merge those subarrays back together in the correct order                                          & n log n                & n log n                \\ \hline
		Linear Search  & Iterate across the array from left-to-right, trying to find the target element.                                                             & n                      & 1                      \\ \hline
		Binary Search  & Given a sorted array, divide and conquer by systematically eliminating half of the remaining elements in the search for the target element. & log n                  & 1                      \\ \hline
		\end{tabular}
		\end{table}
	\end{Topic}
	\begin{Topic}[Recursion]
	Recursion is a divide-and-conquer approach to problem solving that leverages solutions
	to smaller problems (for which solutions are axiomatic or much easier to determine)
	to help in determining the solution to a larger one. \newline
	Draw Sam's machine/diagram on this page! \newline \newline \newline

	Each recursive procedure has two pieces: \newline
	1. Base case (where recursion stops, the simple solution is given) \newline
	2. Recursive case (where the problem is made a little bit smaller but another
	functioncall is made and another frame is placed onto the stack) \newline	\newline
	Example of recursion: Code out factorial in blank space below \newline \newline \newline \newline \newline \newline \newline \newline \newline \newline \newline

	Recursion is great! Howerver, it has downsides, especially given how memory-intensive it can be to employ.
	And while a recursive algorithm is not always required, it frequently looks much
	more beautiful and (though recursion is not itself a simple concept), a recursive
	implementation usually looks much simpler once coded.
	\end{Topic}
\end{spacing}
\end{document}
