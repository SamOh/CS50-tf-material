\documentclass{article}
%%%%%%%%%%%%%%%%%%%%%%%%%%%%%%%%%%%%%%%%%%%%%%%%%%%%%%%%%%%%%
% Lecture Specific Information to Fill Out
%%%%%%%%%%%%%%%%%%%%%%%%%%%%%%%%%%%%%%%%%%%%%%%%%%%%%%%%%%%%%
\newcommand{\LectureTitle}{Week 6: Midterm Review}
%\newcommand{\LectureDate}{\today}
\newcommand{\LectureDate}{October\ 11,\ 2016}
\newcommand{\LectureClassName}{CS\ 50}
\newcommand{\LatexerName}{Samuel\ Oh}
%%%%%%%%%%%%%%%%%%%%%%%%%%%%%%%%%%%%%%%%%%%%%%%%%%%%%%%%%%%%%

% Change "article" to "report" to get rid of page number on title page
\usepackage{amsmath,amsfonts,amsthm,amssymb}
\usepackage{setspace}
\usepackage{Tabbing}
\usepackage{fancyhdr}
\usepackage{hyperref}
\usepackage{lastpage}
\usepackage{extramarks}
\usepackage{chngpage}
\usepackage{soul,color}
\usepackage{graphicx,float,wrapfig}
\usepackage{afterpage}
\usepackage{abstract}
\usepackage{tikz}
\usepackage{ulem}
\let\underbar\uline
% In case you need to adjust margins:
\topmargin=-0.45in
\evensidemargin=0in
\oddsidemargin=0in
\textwidth=6.5in
\textheight=9.0in
\headsep=0.25in

% Setup the header and footer
\pagestyle{fancy}
\lhead{\LatexerName}
\chead{\LectureClassName: \LectureTitle}
\rhead{\LectureDate}
\lfoot{\lastxmark}
\cfoot{}
\rfoot{Page\ \thepage\ of\ \pageref{LastPage}}
\renewcommand\headrulewidth{0.4pt}
\renewcommand\footrulewidth{0.4pt}

%%%%%%%%%%%%%%%%%%%%%%%%%%%%%%%%%%%%%%%%%%%%%%%%%%%%%%%%%%%%%
% Some tools
\newcommand{\enterTopicHeader}[1]{\nobreak\extramarks{#1}{#1 continued on next page\ldots}\nobreak
                                    \nobreak\extramarks{#1 (continued)}{#1 continued on next page\ldots}\nobreak}
\newcommand{\exitTopicHeader}[1]{\nobreak\extramarks{#1 (continued)}{#1 continued on next page\ldots}\nobreak
                                   \nobreak\extramarks{#1}{}\nobreak}

\newlength{\labelLength}
\newcommand{\labelAnswer}[2]
  {\settowidth{\labelLength}{#1}
   \addtolength{\labelLength}{0.25in}
   \changetext{}{-\labelLength}{}{}{}
   \noindent\fbox{\begin{minipage}[c]{\columnwidth}#2\end{minipage}}
   \marginpar{\fbox{#1}}

   % We put the blank space above in order to make sure this
   % \marginpar gets correctly placed.
   \changetext{}{+\labelLength}{}{}{}}

\setcounter{secnumdepth}{0}
\newcommand{\TopicName}{}
\newcounter{TopicCounter}
\newenvironment{Topic}[1][Problem \arabic{TopicCounter}]
  {\stepcounter{TopicCounter}
   \renewcommand{\TopicName}{#1}
   \section{\TopicName}
   \enterTopicHeader{\TopicName}}
  {\exitTopicHeader{\TopicName}}

\setcounter{secnumdepth}{0}
\newcommand{\ExampleSectionName}{}
\newcounter{ExampleSectionCounter}[TopicCounter]
\newenvironment{ExampleSection}[1][Example \arabic{ExampleSectionCounter}]
  {\stepcounter{ExampleSectionCounter}
   \renewcommand{\ExampleSectionName}{#1}
   \section{\ExampleSectionName}
   \enterTopicHeader{\ExampleSectionName}}
  {\exitTopicHeader{\ExampleSectionName}}

\setcounter{secnumdepth}{0}
\newcounter{ExampleBoxCounter}[TopicCounter]
\newcommand{\examplebox}[1]
  {
  % We put this space here to make sure we're disconnected from the previous
   % passage
   \stepcounter{ExampleBoxCounter}
   \noindent\fbox{\begin{minipage}[c]{\columnwidth}#1\end{minipage}}\enterTopicHeader{\ExampleSectionName}\exitTopicHeader{\ExampleSectionName}\marginpar{\fbox{\#\arabic{ExampleBoxCounter}}}
   % We put the blank space above in order to make sure this
   % \marginpar gets correctly placed.
   \vskip10pt
   }

\renewcommand{\contentsname}{{\normalsize Topics Covered}}
\renewcommand{\abstractname}{\LectureTitle\ Summary}
\renewcommand{\absnamepos}{flushleft}

\theoremstyle{definition}
\newtheorem*{dfn}{Definition}
%----------------------------------------------------------------------------------------
%	CUSTOM COMMANDS DUE TO LAZINESS
%----------------------------------------------------------------------------------------

%--------------------------------------------
% Numbers cuz reasons
\newcommand{\C}{{\mathbb C}}
\newcommand{\Q}{{\mathbb Q}}
\newcommand{\R}{{\mathbb R}}
\newcommand{\N}{{\mathbb N}}
\newcommand{\Z}{{\mathbb Z}}
\newcommand{\Sp}{{\mathbb S}}
\newcommand{\A}{{\mathbb A}}

%--------------------------------------------
% Symbols are long
\newcommand{\ph}{{\Phi}}
\newcommand{\vp}{{\varphi}}
\newcommand{\ra}{{\rightarrow}}
\newcommand{\mt}{{\mapsto}}
% Independence
\newcommand\ind{\protect\mathpalette{\protect\independenT}{\perp}}
\def\independent#1#2{\mathrel{\rlap{$#1#2$}\mkern2mu{#1#2}}}

%--------------------------------------------
% Partial derivatives are hard
\newcommand{\p}[2]{\frac{\partial #1}{\partial #2}}
\newcommand{\pp}[2]{\frac{\partial^2 {#1}} {\partial {#2} ^2}}
\newcommand{\pmix}[3]{\frac{\partial^2 {#1}}{\partial {#2}\, \partial {#3}}}

%----------------------------------------------------------------------------------------
%	COMMANDS THAT I NEVER REMEMBER LEL
%----------------------------------------------------------------------------------------

% Integrals
% \int_{a}^{b} [insert integrand] [insert differential]

% Double integral
% \iint_{Values being integrated over} [insert integrand] [\, differential] [\, differential]

% Multi integral
% \idotsint_{Values} [insert integrand f(x, \dots, x_n)] [\, differential \dots differential]

% Summation
% \sum_{n = start value}^{end value} [function]

% Limits
% \lim_{x \to value} [function]

% I.I.D.
% $\overset{\text{i.i.d.}}{\sim}$

% Gradient
% \nabla (\nabla deez nuts)

% Cases
% \begin{cases}
%	{case 1} & \text{if \,} {condition} \\
%	{case 2} & \text{if \,} {condition} \\
% \end{cases}

\begin{document}
\begin{spacing}{1.1}
\newpage

\tableofcontents
\addtocontents{toc}{~\hfill\textbf{Page}\par}
\vskip10pt
\hrule
\vskip10pt

% When topics are long, it may be desirable to put a \newpage or a
% \clearpage before each Topic environment
% \newpage
\begin{Topic}[CS50 Study]
	This will be a huge help for you before and during your test!
\end{Topic}
% Use \Roman{TopicCounter} for each new topic so it increments in the table of contents
% Use \begin{ExampleSection} and \examplebox{} as is necessary
\begin{Topic}[Almost Exhaustive List of Topics]
Below is an attempt at an exhaustive list of topics that can be covered in the midterm. \newline \newline
Week 0

Scratch
\newline
Week 1

Variables and data types

Conditions

Boolean expressions

Loops

Compiling

Overflow

Floating-point imprecision \newline
Week 2

Debugging

Command-line arguments

Arrays

Functions

Scope \newline
Week 3

Asymptotic notation

Linear search

Binary search

Bubble sort

Selection sort

Insertion sort

Merge sort

Recursion \newline
Week 4

Redirection and pipes

File I/O

Memory management

Heap

Stack

Pointers

Structures \newline
Week 5

Linked lists

Hash tables

Tries

Stacks

Queues

Huffman coding
\end{Topic}
\begin{Topic}[Tips to help with solving problems]
You will have access to study.cs50.net, section notes, lecture notes, and past quizzes/solutions
so you should definitely take advantage of this and use them! \newline \newline
1. Pseudocode \newline
If you can write out a pseudocode example of your solution, chances are you will get
the question correct, and, at the very least, you will get partial credit for understanding
the logic behind it. \newline \newline
2. Pay attention to key words \newline
Usually in the quiz quesions, they will include guiding key words, such as 'dynamically
allocated memory' (which alludes to using malloc), or 'if A is not B, then C' (which alludes to
using an if statement to check that A != to B). \newline \newline
Remember, the question gives you a lot of clues on how to implement your solution. Pay close
attention to these details as they can help you think of the correct answer!
\end{Topic}
\begin{Topic}[More Practice Questions!]
The problems shown below are from previous practice exams and can be found on the cs50 website. These specific
problems were chosen because the best reflect the types of questions that will appear on this year's exam. The exam
itself will consist of 10 longer-form questions of the types shown below. I highly recommend going through these questions!
Solutions are also posted on the CS50 website. Good luck! \newline \newline
2015 Quiz 0

8, 17, 18, 23 \newline \newline
2014 Quiz 1

4 \newline \newline
2014 Quiz 0

27 \newline \newline
2013 Quiz 1

0, 1, 2, 12, 14 (not the actual question, but the idea), 17, 20, 22 \newline \newline
2013 Quiz 0

8, 21, 22
\end{Topic}
\end{spacing}
\end{document}
