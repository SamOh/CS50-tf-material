\documentclass{article}
%%%%%%%%%%%%%%%%%%%%%%%%%%%%%%%%%%%%%%%%%%%%%%%%%%%%%%%%%%%%%
% Lecture Specific Information to Fill Out
%%%%%%%%%%%%%%%%%%%%%%%%%%%%%%%%%%%%%%%%%%%%%%%%%%%%%%%%%%%%%
\newcommand{\LectureTitle}{Week 9}
%\newcommand{\LectureDate}{\today}
\newcommand{\LectureDate}{November\ 1,\ 2016}
\newcommand{\LectureClassName}{CS\ 50}
\newcommand{\LatexerName}{Samuel\ Oh}
%%%%%%%%%%%%%%%%%%%%%%%%%%%%%%%%%%%%%%%%%%%%%%%%%%%%%%%%%%%%%

% Change "article" to "report" to get rid of page number on title page
\usepackage{amsmath,amsfonts,amsthm,amssymb}
\usepackage{setspace}
\usepackage{Tabbing}
\usepackage{fancyhdr}
\usepackage{hyperref}
\usepackage{lastpage}
\usepackage{extramarks}
\usepackage{chngpage}
\usepackage{soul,color}
\usepackage{graphicx,float,wrapfig}
\usepackage{afterpage}
\usepackage{abstract}
\usepackage{tikz}
\usepackage{ulem}
\let\underbar\uline
% In case you need to adjust margins:
\topmargin=-0.45in
\evensidemargin=0in
\oddsidemargin=0in
\textwidth=6.5in
\textheight=9.0in
\headsep=0.25in

% Setup the header and footer
\pagestyle{fancy}
\lhead{\LatexerName}
\chead{\LectureClassName: \LectureTitle}
\rhead{\LectureDate}
\lfoot{\lastxmark}
\cfoot{}
\rfoot{Page\ \thepage\ of\ \pageref{LastPage}}
\renewcommand\headrulewidth{0.4pt}
\renewcommand\footrulewidth{0.4pt}

%%%%%%%%%%%%%%%%%%%%%%%%%%%%%%%%%%%%%%%%%%%%%%%%%%%%%%%%%%%%%
% Some tools
\newcommand{\enterTopicHeader}[1]{\nobreak\extramarks{#1}{#1 continued on next page\ldots}\nobreak
                                    \nobreak\extramarks{#1 (continued)}{#1 continued on next page\ldots}\nobreak}
\newcommand{\exitTopicHeader}[1]{\nobreak\extramarks{#1 (continued)}{#1 continued on next page\ldots}\nobreak
                                   \nobreak\extramarks{#1}{}\nobreak}

\newlength{\labelLength}
\newcommand{\labelAnswer}[2]
  {\settowidth{\labelLength}{#1}
   \addtolength{\labelLength}{0.25in}
   \changetext{}{-\labelLength}{}{}{}
   \noindent\fbox{\begin{minipage}[c]{\columnwidth}#2\end{minipage}}
   \marginpar{\fbox{#1}}

   % We put the blank space above in order to make sure this
   % \marginpar gets correctly placed.
   \changetext{}{+\labelLength}{}{}{}}

\setcounter{secnumdepth}{0}
\newcommand{\TopicName}{}
\newcounter{TopicCounter}
\newenvironment{Topic}[1][Problem \arabic{TopicCounter}]
  {\stepcounter{TopicCounter}
   \renewcommand{\TopicName}{#1}
   \section{\TopicName}
   \enterTopicHeader{\TopicName}}
  {\exitTopicHeader{\TopicName}}

\setcounter{secnumdepth}{0}
\newcommand{\ExampleSectionName}{}
\newcounter{ExampleSectionCounter}[TopicCounter]
\newenvironment{ExampleSection}[1][Example \arabic{ExampleSectionCounter}]
  {\stepcounter{ExampleSectionCounter}
   \renewcommand{\ExampleSectionName}{#1}
   \section{\ExampleSectionName}
   \enterTopicHeader{\ExampleSectionName}}
  {\exitTopicHeader{\ExampleSectionName}}

\setcounter{secnumdepth}{0}
\newcounter{ExampleBoxCounter}[TopicCounter]
\newcommand{\examplebox}[1]
  {
  % We put this space here to make sure we're disconnected from the previous
   % passage
   \stepcounter{ExampleBoxCounter}
   \noindent\fbox{\begin{minipage}[c]{\columnwidth}#1\end{minipage}}\enterTopicHeader{\ExampleSectionName}\exitTopicHeader{\ExampleSectionName}\marginpar{\fbox{\#\arabic{ExampleBoxCounter}}}
   % We put the blank space above in order to make sure this
   % \marginpar gets correctly placed.
   \vskip10pt
   }

\renewcommand{\contentsname}{{\normalsize Topics Covered}}
\renewcommand{\abstractname}{\LectureTitle\ Summary}
\renewcommand{\absnamepos}{flushleft}

\theoremstyle{definition}
\newtheorem*{dfn}{Definition}
%----------------------------------------------------------------------------------------
%	CUSTOM COMMANDS DUE TO LAZINESS
%----------------------------------------------------------------------------------------

%--------------------------------------------
% Numbers cuz reasons
\newcommand{\C}{{\mathbb C}}
\newcommand{\Q}{{\mathbb Q}}
\newcommand{\R}{{\mathbb R}}
\newcommand{\N}{{\mathbb N}}
\newcommand{\Z}{{\mathbb Z}}
\newcommand{\Sp}{{\mathbb S}}
\newcommand{\A}{{\mathbb A}}

%--------------------------------------------
% Symbols are long
\newcommand{\ph}{{\Phi}}
\newcommand{\vp}{{\varphi}}
\newcommand{\ra}{{\rightarrow}}
\newcommand{\mt}{{\mapsto}}
% Independence
\newcommand\ind{\protect\mathpalette{\protect\independenT}{\perp}}
\def\independent#1#2{\mathrel{\rlap{$#1#2$}\mkern2mu{#1#2}}}

%--------------------------------------------
% Partial derivatives are hard
\newcommand{\p}[2]{\frac{\partial #1}{\partial #2}}
\newcommand{\pp}[2]{\frac{\partial^2 {#1}} {\partial {#2} ^2}}
\newcommand{\pmix}[3]{\frac{\partial^2 {#1}}{\partial {#2}\, \partial {#3}}}

%----------------------------------------------------------------------------------------
%	COMMANDS THAT I NEVER REMEMBER LEL
%----------------------------------------------------------------------------------------

% Integrals
% \int_{a}^{b} [insert integrand] [insert differential]

% Double integral
% \iint_{Values being integrated over} [insert integrand] [\, differential] [\, differential]

% Multi integral
% \idotsint_{Values} [insert integrand f(x, \dots, x_n)] [\, differential \dots differential]

% Summation
% \sum_{n = start value}^{end value} [function]

% Limits
% \lim_{x \to value} [function]

% I.I.D.
% $\overset{\text{i.i.d.}}{\sim}$

% Gradient
% \nabla (\nabla deez nuts)

% Cases
% \begin{cases}
%	{case 1} & \text{if \,} {condition} \\
%	{case 2} & \text{if \,} {condition} \\
% \end{cases}

\begin{document}
\begin{spacing}{1.1}
\newpage

\tableofcontents
\addtocontents{toc}{~\hfill\textbf{Page}\par}
\vskip10pt
\hrule
\vskip10pt

% When topics are long, it may be desirable to put a \newpage or a
% \clearpage before each Topic environment
% \newpage
% Use \Roman{TopicCounter} for each new topic so it increments in the table of contents
% Use \begin{ExampleSection} and \examplebox{} as is necessary
\begin{Topic}[Decorators]
What is a decorator?\\\\
A decorator in Python is a function that modifies the behavior of other functions,
typically applying some extra functionality.\\
Let's look at an example and determine what it does (this example also shows the syntax of the @
to signify the decorator)
\begin{verbatim}
def override(func):
    def incr():
        return func() + 1
    return incr

@override
def one():
    return 1

print(one())
# prints 2, since its behavior has been overridden (its return value has been increased by 1, thanks to `override()`)
\end{verbatim}
For the most part (in the next pset) you will be using the $@app.route()$ decorator
to specify the URL associated with a function and which methods (HTTP GET, HTTP POST) apply.
\end{Topic}
\begin{Topic}[Flask]
Flask is a Python-based web microframework that automates the process of building simple web apps.

Applications are typically written in a file called application.py, and the behavior of the entire Flask site springs from there.

The simplest Flask app might look something like this:
\begin{verbatim}
from flask import Flask

app = Flask(__name__)

@app.route("/")
def index():
    return "You are at the index!"
\end{verbatim}
What does this code do? What does the $@app.route$ do in this case? To add more url app
routes just add more to this application.py file!
\end{Topic}
\begin{Topic}[Jinja]
Jinja is a Python inspired templating language. It basically makes it so that you don't
have to rewrite html code (which is tedious and a waste of time). This allows you to
have things like a base html file that is taken in and modified (used as a template) by
other pages in your code. Very convenient stuff.

Basically, you define a simple layout.html or equivalent, leaving some parts as
"fill-ins" using Jinja’s syntax (this is your page template), and then other .html
files "extend" this template, filling in those gaps left in the template, sort of
like a code-based mad libs!

You don't have to know much about this, but if you see different syntax in html files, this is
that. This is very convenient to use when you are actually developing on the web too.
Saves a lot of time and helps make less mistakes.
\end{Topic}
\begin{Topic}[SQL]
What is SQL? It is a programming language that is used basically only to manage data in a
database. It stands for Strucutred Query Language. The 4 SQL commands you'll need to know
are the ones below.
\begin{verbatim}
1. Update: Update data in a database table

# update table, changing values in particular columns
UPDATE table SET col1 = val1, col2 = val2, ...

# update table, changing col1 to val1 where "name" equals "identifier"
UPDATE table SET col1 = val1 WHERE identifier = "name"


2. INSERT INTO: Insert certain values into a table

# insert into table the row of values
INSERT INTO table VALUES values

# insert into table under columns col1 & col2, val1 & val2
INSERT INTO table (col1, col2) VALUES (val1, val2)


3. SELECT: Select values to view

# select a column from table to compare/ view
SELECT col FROM table WHERE col = "identifier"

# select all columns from a table
SELECT * FROM table
(generally the * in programming means everything. Fun fact.)



4. DELETE: Delete from table

# delete a row from table
DELETE FROM table WHERE col = "identifier"
\end{verbatim}
\end{Topic}
\end{spacing}
\end{document}
