\documentclass{article}
%%%%%%%%%%%%%%%%%%%%%%%%%%%%%%%%%%%%%%%%%%%%%%%%%%%%%%%%%%%%%
% Lecture Specific Information to Fill Out
%%%%%%%%%%%%%%%%%%%%%%%%%%%%%%%%%%%%%%%%%%%%%%%%%%%%%%%%%%%%%
\newcommand{\LectureTitle}{Week 8}
%\newcommand{\LectureDate}{\today}
\newcommand{\LectureDate}{October\ 23,\ 2016}
\newcommand{\LectureClassName}{CS\ 50}
\newcommand{\LatexerName}{Samuel\ Oh}
%%%%%%%%%%%%%%%%%%%%%%%%%%%%%%%%%%%%%%%%%%%%%%%%%%%%%%%%%%%%%

% Change "article" to "report" to get rid of page number on title page
\usepackage{amsmath,amsfonts,amsthm,amssymb}
\usepackage{setspace}
\usepackage{Tabbing}
\usepackage{fancyhdr}
\usepackage{hyperref}
\usepackage{lastpage}
\usepackage{extramarks}
\usepackage{chngpage}
\usepackage{soul,color}
\usepackage{graphicx,float,wrapfig}
\usepackage{afterpage}
\usepackage{abstract}
\usepackage{tikz}
\usepackage{ulem}
\let\underbar\uline
% In case you need to adjust margins:
\topmargin=-0.45in
\evensidemargin=0in
\oddsidemargin=0in
\textwidth=6.5in
\textheight=9.0in
\headsep=0.25in

% Setup the header and footer
\pagestyle{fancy}
\lhead{\LatexerName}
\chead{\LectureClassName: \LectureTitle}
\rhead{\LectureDate}
\lfoot{\lastxmark}
\cfoot{}
\rfoot{Page\ \thepage\ of\ \pageref{LastPage}}
\renewcommand\headrulewidth{0.4pt}
\renewcommand\footrulewidth{0.4pt}

%%%%%%%%%%%%%%%%%%%%%%%%%%%%%%%%%%%%%%%%%%%%%%%%%%%%%%%%%%%%%
% Some tools
\newcommand{\enterTopicHeader}[1]{\nobreak\extramarks{#1}{#1 continued on next page\ldots}\nobreak
                                    \nobreak\extramarks{#1 (continued)}{#1 continued on next page\ldots}\nobreak}
\newcommand{\exitTopicHeader}[1]{\nobreak\extramarks{#1 (continued)}{#1 continued on next page\ldots}\nobreak
                                   \nobreak\extramarks{#1}{}\nobreak}

\newlength{\labelLength}
\newcommand{\labelAnswer}[2]
  {\settowidth{\labelLength}{#1}
   \addtolength{\labelLength}{0.25in}
   \changetext{}{-\labelLength}{}{}{}
   \noindent\fbox{\begin{minipage}[c]{\columnwidth}#2\end{minipage}}
   \marginpar{\fbox{#1}}

   % We put the blank space above in order to make sure this
   % \marginpar gets correctly placed.
   \changetext{}{+\labelLength}{}{}{}}

\setcounter{secnumdepth}{0}
\newcommand{\TopicName}{}
\newcounter{TopicCounter}
\newenvironment{Topic}[1][Problem \arabic{TopicCounter}]
  {\stepcounter{TopicCounter}
   \renewcommand{\TopicName}{#1}
   \section{\TopicName}
   \enterTopicHeader{\TopicName}}
  {\exitTopicHeader{\TopicName}}

\setcounter{secnumdepth}{0}
\newcommand{\ExampleSectionName}{}
\newcounter{ExampleSectionCounter}[TopicCounter]
\newenvironment{ExampleSection}[1][Example \arabic{ExampleSectionCounter}]
  {\stepcounter{ExampleSectionCounter}
   \renewcommand{\ExampleSectionName}{#1}
   \section{\ExampleSectionName}
   \enterTopicHeader{\ExampleSectionName}}
  {\exitTopicHeader{\ExampleSectionName}}

\setcounter{secnumdepth}{0}
\newcounter{ExampleBoxCounter}[TopicCounter]
\newcommand{\examplebox}[1]
  {
  % We put this space here to make sure we're disconnected from the previous
   % passage
   \stepcounter{ExampleBoxCounter}
   \noindent\fbox{\begin{minipage}[c]{\columnwidth}#1\end{minipage}}\enterTopicHeader{\ExampleSectionName}\exitTopicHeader{\ExampleSectionName}\marginpar{\fbox{\#\arabic{ExampleBoxCounter}}}
   % We put the blank space above in order to make sure this
   % \marginpar gets correctly placed.
   \vskip10pt
   }

\renewcommand{\contentsname}{{\normalsize Topics Covered}}
\renewcommand{\abstractname}{\LectureTitle\ Summary}
\renewcommand{\absnamepos}{flushleft}

\theoremstyle{definition}
\newtheorem*{dfn}{Definition}
%----------------------------------------------------------------------------------------
%	CUSTOM COMMANDS DUE TO LAZINESS
%----------------------------------------------------------------------------------------

%--------------------------------------------
% Numbers cuz reasons
\newcommand{\C}{{\mathbb C}}
\newcommand{\Q}{{\mathbb Q}}
\newcommand{\R}{{\mathbb R}}
\newcommand{\N}{{\mathbb N}}
\newcommand{\Z}{{\mathbb Z}}
\newcommand{\Sp}{{\mathbb S}}
\newcommand{\A}{{\mathbb A}}

%--------------------------------------------
% Symbols are long
\newcommand{\ph}{{\Phi}}
\newcommand{\vp}{{\varphi}}
\newcommand{\ra}{{\rightarrow}}
\newcommand{\mt}{{\mapsto}}
% Independence
\newcommand\ind{\protect\mathpalette{\protect\independenT}{\perp}}
\def\independent#1#2{\mathrel{\rlap{$#1#2$}\mkern2mu{#1#2}}}

%--------------------------------------------
% Partial derivatives are hard
\newcommand{\p}[2]{\frac{\partial #1}{\partial #2}}
\newcommand{\pp}[2]{\frac{\partial^2 {#1}} {\partial {#2} ^2}}
\newcommand{\pmix}[3]{\frac{\partial^2 {#1}}{\partial {#2}\, \partial {#3}}}

%----------------------------------------------------------------------------------------
%	COMMANDS THAT I NEVER REMEMBER LEL
%----------------------------------------------------------------------------------------

% Integrals
% \int_{a}^{b} [insert integrand] [insert differential]

% Double integral
% \iint_{Values being integrated over} [insert integrand] [\, differential] [\, differential]

% Multi integral
% \idotsint_{Values} [insert integrand f(x, \dots, x_n)] [\, differential \dots differential]

% Summation
% \sum_{n = start value}^{end value} [function]

% Limits
% \lim_{x \to value} [function]

% I.I.D.
% $\overset{\text{i.i.d.}}{\sim}$

% Gradient
% \nabla (\nabla deez nuts)

% Cases
% \begin{cases}
%	{case 1} & \text{if \,} {condition} \\
%	{case 2} & \text{if \,} {condition} \\
% \end{cases}

\begin{document}
\begin{spacing}{1.1}
\newpage

\tableofcontents
\addtocontents{toc}{~\hfill\textbf{Page}\par}
\vskip10pt
\hrule
\vskip10pt

% When topics are long, it may be desirable to put a \newpage or a
% \clearpage before each Topic environment
% \newpage
% Use \Roman{TopicCounter} for each new topic so it increments in the table of contents
% Use \begin{ExampleSection} and \examplebox{} as is necessary
\begin{Topic}[Python]
Why Python? What is different about Python? Python is currently used in a lot of web development
and is a very powerful, higher-level language(higher-level in the sense of abstraction).
You'll find that the syntax is much more intuitive and this language is easier to use. I hope
you like it (I really like Python and use it for most of my coding projects)\\\\
To start writing python code, open up a file with .py as the extension. Also, with python, you
don't have to compile your code before running! Now you to run your code, just write:\\
python3 example.py
\end{Topic}
\begin{Topic}[Variables]
There are two main differences\\
1. No type declarations necessary\\
2. To declare a variable, you need to initialize it.\\
Let's look at some examples in code!
\begin{verbatim}
#include <cs50.h>
#include <stdio.h>

int main(void)
{
    string s = get_string();
    printf("hello, %s\n", s);
}

import cs50

s = cs50.get_string()
print("hello, {}".format(s))

\end{verbatim}
\textbf{KEY TAKEAWAYS: No curly braces, no semicolons, no type declarations, variables must
be initialized instead of declared alone.}
\end{Topic}
\begin{Topic}[Conditionals]
The same conditionals in C exist in python as well! (if/else, loops, etc.) and
boolean operators/comparators are different too. So, instead of using $||$ like you did
in C, you would now just say "or". And instead of $\&\&$, you just use "and". Same thing with
$!$ and "not".\\\\
Now lets look at some key code examples to help move into the python world.\\\\
I can send the code out along with the files if you would find that helpful!
Code is in file python/2-conditions and python/3-logical\\\\
\textbf{KEY TAKEAWAYS: instead of encapsulating code within curly braces, use a colon and indentation.
Also, conditions don't have to be in parenthesis}
\end{Topic}
\begin{Topic}[Loops]
While loops and for loops exist in python. Lets look at some examples to help us understand/see
the key differences.\\
\begin{verbatim}
while height > 0:
  # code here

while (height > 0)
{
  // code here
}
\end{verbatim} Here, notice the differences in comments, strucutre, code. Let's look at some more complex
while loop examples in the IDE (I can send these out after too)\\
\begin{verbatim}
for i in range(0,height):
  # use i to do something

for (int i = 0; i < height; i++)
{
  # use i to do something
}
\end{verbatim} Notice again, the changes in syntax. Tabs make a difference, comments are now used with
the $\#$ symbol instead of //. Let's look at more complex for loops!
\textbf{KEY TAKEAWAYS: Understand how a while loop and a for loop differs between c and python.}
\end{Topic}
\begin{Topic}[Arrays/Lists]
Python arrays (more appropriately known as lists) are notfixed in size; they can
grow or shrink as needed, and you can always tack extra elements onto your array
and splice things in and out easily. Lists in python are a lot better/more powerful
than arrays were in C!\\\\
Declaring a list:
\begin{verbatim}
nums = []
nums = [1,2,3,4]
nums.append(5)
What is nums now?

nums.insert(1, 5)
What is nums now?

a = [2,3]
b = a + nums
What is b now?

\end{verbatim}
\textbf{KEY TAKEAWAYS: know how to manipulate lists!}
\end{Topic}
\begin{Topic}[Functions]
Function declaration is very similar to C. Lets look at a quick example:\\
\begin{verbatim}
def square(x):
  x * x

square(2)
what is returned?

\end{verbatim}
\textbf{KEY TAKEAWAYS: you don't need to specify return type, use def keyword, and inputs
are in the parenthesis}
\end{Topic}
\begin{Topic}[MVC]
Model View Controller (MVC) is an important web software paradigm to help us organize our code when
developing on the web. Lets take a look at what CS50.study has to say about this!
\end{Topic}
\begin{Topic}[Cool, fun python examples!]
Look at some cool, fun code examples/things you can do with python!
\end{Topic}
\end{spacing}
\end{document}
